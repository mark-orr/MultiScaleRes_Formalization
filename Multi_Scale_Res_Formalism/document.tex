\documentclass{article}
\title{Formalization of the Multi-Resolution Thesis and the Reciprocal Constraints Paradigm for Understanding Social Systems}
\author{Mark G. Orr}

\begin{document}
\maketitle
\section{General Structure of the Reciprocal Constraints Paradigm}
The RCP is a methodological approach for thinking about how to develop understanding of social systems, and ultimately provide useable simulations of such.  It's value does not lie in precise presecription, but in laying a foundation for fruitful social simulation and an understanding of the implications for theory and models across levels of scale.  Figure \ref{RCP_diagram} shows the four primary components of the RCP: a cognitive system, an upward-scaling constraint, a social system, and a downward-scaling constraint.  Figure \ref{RCP_diagram} captures the potential for integrating neurophysiolgical considerations when appropriate; these may prove as essential for some social systems.

A central assumption in the RCP is that cognitive systems and definitions of agent behaviors in social systems are representing human information-processing capacities that can be described as mathematical functions. \cite{van Rooij, 2008}\footnote{This is equivalent to Marr's computational level; we will use Marr's computational, algorithmic, and implementation levels of description\cite{Marr,1981} throughout this paper.}. From this assumption we can define a cognitive system as $\psi_{ct}: I_{ct} \rightarrow \psi_{ct}(i)$ where $I_{ct}$ is the set of allowable inputs and $\psi_{ct}(i)$ is the output; social systems, then, have a corresponding agent definition as $\phi_{at}: I_{at} \rightarrow \phi_{at}(i)$ where $I_{at}$ is the set of allowable inputs and $\phi_{at}(i)$ is the output for an agent\footnote{Social and cognitive systems may define parameters regarding variability among a set of agents; this is not reflected here.}.   Further, we assume that defining social systems and cognitive models as having an abstract set of first principles $S$ and $C$, repectively such that $s \in S$ and $c \in C$ for $\phi_{at}$ and $\psi_{ct}$. Notice that $S$ and $C$ imply somthing about the inputs for both levels of scale as well as each's functional mapping.  $S$ and $C$ are theoretical entities that determine what is allowable; these will be defined more precisely below. The interpretation of the upward- and downward-scaling constraints we use is that each level of scale \textit{inhierits} properties from the adjacent level.\footnote{In this treatment, we only have two levels, but this is not necessarily the case.}  We will expand on constraints below. 

\subsection{Constraints, Inhieritence \& First Principles}
The notion of constraints and their nature is unbounded, in principle, except to the extent that they are tighly bounded to the first principles $C$ and $S$.    Here, we provide some examples that seem natural to a first approximation; these are not meant to be exhaustive or restrictive.  Further, the nature of constraints may likely be quite different when comparing upward- to downward-constraints.
 
The empirical argument.  A fundamental component of constraints is that they are bound up

The algorithmic considerations; where does this fit in?

\subsubsection{Upward-Constraints}
The constraints one type of inhieritance from the cognitive to social would be to consider the degree to which $\phi_{at}$ the $s$ in $S$ compares to $C$; in other words, to what degree does $s$ respect first principles of human cognition.  These are approaches that may not necessitate direct simulation, but would require a disciplined appraoch fo such a comparison. 

\subsubsection{Downward-Constraints}
but by example, many social systems will define information flow among agents as a graph $G$ where $V(G)$ are the agents and $E(G)$ define the information channels among agents. 


\subsubsection{Automation and Systems of Constraints}
The notion of generating a fixed system of constraints that can automate the construction of social systems.



\end{document}