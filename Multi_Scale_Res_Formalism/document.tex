\documentclass{article}
\title{Formalization of the Multi-Resolution Thesis and the Reciprocal Constraints Paradigm for Understanding Social Systems}
\author{Mark G. Orr}

\begin{document}
\maketitle
The RCP is a methodological approach for thinking about how to construct understanding of social systems.  It's value does not lie in precise presecription, but in laying the foundation for fruitful social simulation and the implications for theory and models across levels of scale.  Figure \ref{figureX} shows the four primary components of the RCP: a cognitive system, an upward-scaling constraint, a social system, and an a downward-scaling constraint. A central assumption in the RCP is that cognitive systems and definitions of agent behaviors in social systems are representing human information-processing capacities that can be described as mathematical functions. \cite{van Rooij, 2008}\footnote{This is equivalent to Marr's computational level; we will use Marr's computational, algorithmic, and implementation levels of description\cite{Marr,1981} throughout this paper.}. From this assumption we can define a cognitive system as $\psi_{ct}: I_{ct} \rightarrow \psi_{ct}(i)$ where $I_{ct}$ is the set of allowable inputs and $\psi_{ct}(i)$ is the output; social systems, then, have a corresponding agent definition as $\phi_{at}: I_{at} \rightarrow \phi_{at}(i)$ where $I_{at}$ is the set of allowable inputs and $\phi_{at}(i)$ is the output for an agent\footnote{Social and cognitive systems may define parameters regarding variability among a set of agents; this is not reflected here.}.  Further, for some social systems, information flow among agents may be defined as a graph $G$ where $V(G)$ are the agents and $E(G)$ define the information channels among agents\footnote{Social simulations, in principle, can define more than one graph and more than one agent type, but for simplicity, we will not include this in our formalization scheme.}. 

Further, we assume that defining social systems and cognitive models as having an abstract set of first principles $S$ and $C$, repectively such that $s \in S$ and $c \in C$ for $\phi_{at}$ and $\psi_{ct}$. Notice that $S$ and $C$ imply somthing about the inputs for both levels of scale as well as each's functional mapping.  $S$ and $C$ are theoretical entities that determine what is allowable; these will be define more precisely below.  

NOTE
 

  

\end{document}