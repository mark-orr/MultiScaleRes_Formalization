\documentclass{article}
\title{Multi-Scale Resolution of Human Social Systems:  A Synergistic Paradigm for Simulating Minds and Society}
\author{Mark G. Orr}

\begin{document}
\maketitle

\section{Introduction}
Recently, we put forth an initial sketch of what we call the Resolution Thesis.  The thesis holds that 1) models of cognition will be improved given constraints from the structure and dynamics of the social systems in which they are supposed to be embedded, and 2) the resolution of social simulations of agents will be improved given constraints from cognitive first principles.  This thesis reflects a variety of motivations, the most obvious being the observation that there is little overlap between the cognitive sciences and the generative social science approach[*REF EPSTEIN, 2008], both of which rely heavily on computer simulation to understand aspects of human systems, albeit different aspects of human systems with respect to scale.  The former focuses almost exclusively on the mind as a scientific object of study for which the lion's share of simulation efforts reflect a normative mind (one mind at a time, non-social or statically social) and the latter emphasizing multiple aspects of social systems, the mind being only one of these aspects, with the effect there is little cross-polination with the vast experimental evidence of how the mind operates or what are the central theoretical entities that compose the mind.  To a large degree, the resolution thesis is a recognition that the interdependence between cognitive and social systems has yet to be leveraged for the purposes of improving our understanding of both, despite some common points of interest.    

The Resolution Thesis can be understood from multiple perspectives.  From the view of cognitive science, the thesis means, broadly, that patterns of organization (e.g., information flow on the internet [REF; FROM SOCIALSIM], clustering of behaviors in a community[REF CHRISTAKIS] at the social and organizational level should inform the specifications of a cognitive model.  In other words, when possible, these patterns should be included as convergent evidence for a theory or model.  Naturally, it is desirable for cognitive models that are implicated in social behavior to have some reasonable explanatory scheme that links facets of the cognitive model to aspects of social organization.   We will address what this means and issues in more detail below but want to point the reader to Anderson's 200X Relevance Thesis as a useful example of reasoning about how cognition may have reasonable implications for social organization.   Furthermore, from the cognitive science perspective, neurophysiological processes are naturally implied and should be considered as a key part of the Resolution Thesis when appropriate [REF STOCCO here].  The reader might notice that it is difficult to think about using social organization as part and parcel of the convergent evidence of a cognitive model without considering the implications of cognition for social behavior.  But that is precisely the point--it is natural, once of the mind to think about how cognition scales, to think about using the degree to which it scales accurately as part of the convergent evidence for the validity of the model.


From the generative social science perspective, the Resolution Thesis means that the representations of agents should be informed closely by cognitive science and relevant neurophysiological considerations.  This runs somewhat counter to the principled adherence to simplification of the internal processing of simulated agents found in this literature, one that, in fact, served to show that complex social dynamics can be driven by simple behavioral rules of agents.  However, more recently, there are efforts in the generative social sciences that acknowledged that closer ties to the psychological and neurophysiological underpinnings of human behavior may yield benefit.  Epstein's neurocognitive approach is a notable effort in this vein[REF, Epstein 2014]; there are other related approaches (e.g. REF 19-22 from BRiMS).  These efforts notwithstanding, there remains a large gap between them and the implementation of models from cognitive science and psychology, not necessarily in principle, but in practice.  

A third and more general view is that the Resolution Thesis is about human systems for which the distinction between neural, cognitive and social levels of scale should be considered in unison.  An understanding of any of these levels of scale is dependent, to some degree, on an understanding of the others.  In effect, the notion of convergent evidence as originating, in part from other levels of scale, applies to all.  The implication is that we should leverage information from different levels of scale in an iterative and synergistic way, if not simultaneously [REF BRiMS paper][MAYBE REF ABDUCTION NAND 

The Resolution Thesis, despite sounding both reasonable and practical at face value, faces opposition from both the cognitive sciences and generative social sciences.  Simon's notion of nearly decomposable systems--that the temporal dynamics of adjacent levels of scale, in most systems, are little correlated--suggests that we can understand well the dynamics at each level of scale independently of the others(*REF, SIM 1962; see P. Anderson, 1972, for similiar argument in phyical systems).  The generative social sciences KISS (keep it simple, stupid) principle is clearly akin to Simon's notion, and is bolstered by the early wins in understanding the behavior of social systems (REF 15-17).   Another related, and possibley purer/stronger argument is that complexity arises in the midst of simplicity (e.g., May, 1976) a notion that is also well reflected in the origins of the generative social sciences (REF 15-17 from BRiMS).  In cognitive science, Newell, in considering the time scale of human behavior, suggested that the social band ($> 10^4$ seconds, representing social systems and organizational behavior) is characterized to be weak in strength in the sense that it may not provide computations in a systematic way (relative to lower temporal bands, e.g., cognitive and neural processes).  


These counter arguments notwithstanding, our working assumption is captured by the collquialism "The proof is in the pudding"  That is, the state of the art in technology, computing and the their tight coupling to the current social mileau affords, we think, the testing of the Resolution Thesis.   To this end, we've developed the \textit{Reciprocal Constraints Paradigm}, to be discussed next.


\section{Components of the Reciprocal Constraints Paradigm}
The RCP is a methodological approach for thinking about how to develop understanding of social systems, and ultimately provide useable simulations of such.  It's value does not lie in precise presecription, but in laying a foundation for fruitful social simulation and an understanding of the implications for theory and models across levels of scale.  Figure \ref{RCP_diagram} shows the four primary components of the RCP: a cognitive system, an upward-scaling constraint, a social system, and a downward-scaling constraint.  Figure \ref{RCP_diagram} captures the potential for integrating neurophysiolgical considerations when appropriate; these may prove as essential for some social systems.

Further, we assume that defining social systems and cognitive models as having an abstract set of first principles $S$ and $C$, repectively such that $s \in S$ and $c \in C$ for $\phi_{at}$ and $\psi_{ct}$. Notice that $S$ and $C$ imply somthing about the inputs for both levels of scale as well as each's functional mapping.  $S$ and $C$ are theoretical entities that determine what is allowable; these will be defined more precisely below. The interpretation of the upward- and downward-scaling constraints we use is that each level of scale \textit{inherits} properties from the adjacent level, in a highly abstract way.\footnote{In this treatment, we only have two levels, but this is not necessarily the case.}  We will expand on constraints below. 

A central assumption in the RCP is that cognitive systems and definitions of agent behaviors in social systems are meant to represent human information-processing capacities that can be described as mathematical functions. \cite{van Rooij, 2008}\footnote{This is equivalent to Marr's computational level; we will use Marr's computational, algorithmic, and implementation levels of description\cite{Marr,1981} throughout this paper.}. Thus, in $C$ we can define a cognitive system as $\psi_{ct}: I_{ct} \rightarrow \psi_{ct}(i)$ where $I_{ct}$ is the set of allowable inputs and $\psi_{ct}(i)$ is the output; in $S$ we have a corresponding agent definition as $\phi_{at}: I_{at} \rightarrow \phi_{at}(i)$ where $I_{at}$ is the set of allowable inputs and $\phi_{at}(i)$ is the output for an agent\footnote{Social and cognitive systems may define parameters regarding variability among a set of agents; this is not reflected here.}.   

The notion of reciprocal constraint, Constraints are embedded in $C$ and $S$ and can reflect both temporal and structural aspects of

A primary example in $C$ is the set of allowable algorithms $A$ such that $a \in C$ that have some grounding in empirical work in cognitive science and psychology.  This is a task complicated by the fact that within cognitive science and psychology there are sometimes strong disagreements concerning what is the right $A$, theroetically speaking, in terms of the task (what is the human analong that the models is representing)\footnote{Issues of computational complexity and human cognitive functions are outside of the scope of this argument at the present time, but should be included in the future\ref{vonRooij}}.  Despite this difficulty, not all algorithms that compute $\psi_{ct}$ are in $C$.  

Primary examples in $S$ are the social structures, channels of information, and dynamics that characterize a social system, much of which is formalized using graph theory/network science.  These constraints in $S$ could directly affect the distribution of the inputs for a cognitive system.  It is important to notice that within $S$ are notions regarding the behavior of agents   

given a social system $S$ does not allow information from certain branches...in other words the social system affects the allowable inputs in a given unit time.  These have some grounding in empirical work on social structure.   but by example, many social systems will define information flow among agents as a graph $G$ where $V(G)$ are the agents and $E(G)$ define the information channels among agents, $\phi_{at}$.  

A fundamental component of constraints, as we see it, is that they are bound up in both theory and empirical/observation work in respect to the discipines that address a particular level of scale.  This makes defining what is $S$ and $C$ precisely more a matter of practice, at least in the initial stages of applications of the \textit{RCP}. 

\subsection{Applying the Reciprocal Constraints Paradigm}   
In practice there are several approaches available for using the RCT, but what unites them is the study of a human social phenomena, either defined at one level of scale or at multiple levels of scale.  Naturally, the first step is to identify a social phenonmena of interest, a task that is inherently tied to one's perspective.  If the perspective is largely in $C$, then the focus would most likely be on understanding the psychological processes, representations, etc.; in other words a normative human\footnote{Individual differences do not preclude the notion of normative human but can be considered a parameter in such a theory}.   Another perspective, largely in $S$, would dictate a concern with the social structures and dynamics of the social system (many humans interacting).  We will address perspectives that treat $C$ and $S$ simultanteously further below.  Next we will provide examples of approaches from these perspectives.

\subsubsection{Largely in $C$}
There exists a large literature on neural and cognitive approaches to understanding human social behavior and social psychology\footnote{Division 8 of the American Psychological Association is dedicated to social psychology } that include a broad range of methods and theoretical orientations.  

\subsubsection{Largely in $S$}
 
The first, and in some ways simplest, is to build a simulation platform that simultaneosly captures essentail aspects of both $S$ and $C$ (e.g., an agent-based model of cognitive agents) and to simulate a social phenomena.  In this case, the upward-constraints refer directly to the substition of  $\psi_{ct}$ for $\phi_{at}$; the downward-constraints refer to the degree to which the simulated social system matches $S$ as defined by the pheonomena of interest.  The downward constraints would thus serve as a signal that would suggest modifications to $\psi_{ct}$.  

The second, and somewhat more complicated, is a suite of approaches that do not require invoking a full simulation system.  For example, given 




The second kind is to abstract some details from full-fledged cognitive models, but adhere to a principle of \textit{Accountable Modeling}\ref{LebiereX}.

Another kind is analysis of The constraints one type of inhieritance from the cognitive to social would be to consider the degree to which $\phi_{at}$ the $s$ in $S$ compares to $C$; in other words, to what degree does $s$ respect first principles of human cognition.  These are approaches that may not necessitate direct simulation, but would require a disciplined appraoch fo such a comparison. 

\subsubsection{Downward-Constraints}
 Kinds(degrees of):
 The first kind is to use a social system to generate distributions of inputs....
 The first kind is to abstract away the details 

\subsubsection{Automation and Systems of Constraints}
The notion of generating a fixed system of constraints that can automate the construction of social systems.  $S$ and $C$ can be encoded as algorithms.  If this is the case, then we could construct an automated system that scans the parameter space (if computable).

\section{Analysis of Prior Work}
Here we provide examples of prior work that employ parts of the RCP.  As we have no known examples of the downward-constraints, we focus on upward-constraints.  These are not exhaustive; we apologize for any work not mentioned.  

\section{An Example of RCT Implemented}
Christian's sim

Note, we may just provide examples of how Stocco's work might be used in our example...not run the actual sims if not have time.

\section{Discussion}
\subsection{Computational Complexity}
Both levels of scale should restrict to functions that are computable in polynomial time.  

\subsection{What Level of Scale for Initiation}
We advocate starting with the cognitive level in principle.  This might mean starting with a social system, implemented on a graph. 

\subsection{Social System Only and Graphical Dynamical Systesm}
There 

\subsection{Statistical and Analytical Models of Social Systems}
Can we apply the RCP to Statistical or Analytic models that do not have implementations in simulation environments?  Is this possible?  If yes, then what are the considerations?

\subsection{Importing Neurophysiology}
Provide background on Stocco's work.

\end{document}