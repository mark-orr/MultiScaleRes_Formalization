\documentclass{article}
\title{Formalization of the Multi-Resolution Thesis and the Reciprocal Constraints Paradigm for Understanding Social Systems}
\author{Mark G. Orr}

\begin{document}
\maketitle
Here we will formalize the reciprocal contraints paradigm, not for the sake of formalization per se but to provide clarity in the definitions of its components.  We begin by assuming that cognitive models are representing human information-processing capacities that can be described as mathematical functions. \cite{van Rooij, 2008}\footnote{This is equivalent to Marr's computational level; we will use Marr's computational, algorithmic, and implementation levels of description\cite{Marr,1981} throughout this paper.}. From this assumption we can define a cognitive model as $\psi_{ct}: I_{ct} \rightarrow \psi_{ct}(i)$ where $I_{ct}$ is the set of allowable inputs and $\psi_{ct}(i)$ is the output.  We also define an agent behavioral model isomprohically as $\phi_{at}: I_{at} \rightarrow \phi_{at}(i)$ where $I_{at}$ is the set of allowable inputs and $\phi_{at}(i)$ is the output for an agent\footnote{Social simulations may define parameters regarding variability among a set of agents; this is not reflected here.}.  For some social systems, information flow among agents may be defined as a graph $G$ where $V(G)$ are the agents and $E(G)$ define the information channels among agents\footnote{Social simulations, in principle, can define more than one graph and more than one agent type, but for simplicity, we will not include this in our formalization scheme.}. Further, we assume that defining social systems and cognitive models as having an abstract set of first principles $S$ and $C$, repectively such that $s \in S$ and $c \in C$ for $\phi_{at}$ and $\psi_{ct}$. Notice that $S$ and $C$ imply somthing about the inputs for both levels of scale as well as each's functional mapping.  $S$ and $C$ are theoretical entities that determine what is allowable; these will be define more precisely below.  

To push our paradigm into a usable form, we must incorporate the algorithmic level of description\cite{Marr}.  In practice, many of the social systems of interest rely on computer simulation.  Thus, the reciprocal constraints paradigm asserts that algorithmic descriptions should be consistent with the related computational descriptions.  It is useful to consider what this means at the cognitive level alone...walk through van Rooij's arguments. Ok good, do it.
 

  

\end{document}